\documentclass[12pt,a4paper]{article}
\usepackage[utf8]{inputenc}
\usepackage[russian]{babel}
\usepackage[left=2.00cm, right=2.00cm, top=2.00cm, bottom=2.00cm]{geometry}
\linespread{1.25}
\usepackage{setspace}
\usepackage{indentfirst}
\setlength{\parindent}{1.25cm}
\let\paragraph\ignorespaces
\usepackage{tabularx}
\usepackage{multirow}
\usepackage{graphicx}


\begin{document}
	
\begin{titlepage}
	
\begin{center}
	\large Университет ИТМО\\[5cm]
	\LARGE Практическая работа №1\\
	\normalsize по дисциплине <<Визуализация и моделирование>>\\[5cm]
\end{center}
\begin{flushright}
		\begin{minipage}{0.6\textwidth}
		\begin{flushleft}
			\large
			\singlespacing 
			\textbf{Автор:} Сулейманов Руслан Имранович
			\textbf{Поток:} 1.1
			\textbf{Группа:} к3241
			\textbf{Факультет:} ИКТ\\
			\textbf{Преподаватель:} Чернышева А.В.
		\end{flushleft}
	\end{minipage}
\end{flushright}

\vfill

\begin{center}
	{\large Санкт-Петербург, \the\year{ г.}}
\end{center}
 
\end{titlepage}
\normalsize

Краткое описание датасета

MNIST - Датасет рукописных цифр. 60 000 тренировочных изображений и 10 000 тестовых изображений.
Датасет состоит из обучающей и тестовой выборки.

Описание данных, хранящихся в датасете 


\begin{tabular}{c|c|c}
 название & какие данные & тип \\
\hline
offset & 4ех значное число & integer \\
type & опичания типа значения & str  \\
value & хранится само значение & 32 bit integer  \\
description & описание значения & str  \\
\end{tabular}


Благодаря этому датосету можно научить свою нейронку распознаванию:

а) мед. почерка.

б) решению капчи.

в) ну или в самом простом варианте, чтению рукаписного текста


\end{document}
